\chapter*{Composición de funciones}
\setcounter{chapter}{1}
\setcounter{section}{0}

\blueBox{Definición}{
    Dadas dos funciones $f: A \rightarrow B$ y $g: B \rightarrow C$, la composición de $f$ y $g$ es la función $g \circ f$ definida por:
    $$g \circ f: A \rightarrow C / (g \circ f)(x) = g(f(x))$$
    No es necesario que el dominio de $g$ sea igual al codominio de $f$, pero sí que la imagen de la función $f$ esté contenida por el dominio de $g$.
    $$Im(f) \subseteq Dom(g)$$
    \tcblower
    \textbf{Ejemplo}: determinar $f \circ g$ y $g \circ f$ de las funciones $f(x) = x^2 + 1$ y $g(x) = -2x + 6$.
    \begin{enumerate}
        \item {Aplicamos la definición:
        $$(f \circ g)(x) = f(g(x))$$
        $$x \rightarrow g(x) \rightarrow f(g(x)) = (g(x))^2 + 1$$
        $$x \rightarrow (-2x+6) \rightarrow (-2x+6)^2 + 1$$
        por ende:
        $$(f \circ g)(x) = 4x^2 -24x +37$$
        }
        \item{Haciendo lo mismo para el otro lado:
        $$g \circ f: \R \rightarrow \R / (g \circ f)(x) = -2x^2 + 4$$
        }
    \end{enumerate}
}

\section{La función identidad}
Dado un conjunto cualquiera, la función que le asigne como imagen el mismo valor que el que recibe se llama función identidad y se denota por $Id_A$.
$$id_A : A \rightarrow A / id_A(x) = x$$
En los números reales la función es:
$$y = x$$
y su gráfica es la bisectriz del primer y tercer cuadrante.

\blueBox{Propiedad}{
    Si $f: A \rightarrow B$ es una función biyectiva entonces se tiene que:
    \begin{enumerate}
        \item $f \circ f^{-1} : B \rightarrow B / (f \circ f^{-1})(x) = x$ es la función $id_B$
        \item $f \circ f^{-1} : A \rightarrow A / (f^{-1} \circ f)(x) = x$ es la función $id_A$
    \end{enumerate}
}
\section{Paridad e imparidad de una función}
Se dice que un dominio $D$ es \textbf{simétrico respecto del origen} si $x \in D$ y su opuesto, $-x$ también pertenece a $D$.

\blueBox{Definición - Función par}{
    Sea una función definida en un conjunto simétrico respecto del origen. $f$ es par si: 
    $$f(-x) = f(x) \,\,\, \forall x \in Dom(f)$$
    Las gráficas de las funciones pares son simétricas respecto al eje de ordenadas.
    \tcblower
    \textbf{Ejemplo}: La función $f(x) = \ln(x^4-16)$ es una función par:\\
    La función está definida si
    $$x^4 - 16 > 0 \Leftrightarrow x^4 > 16 \Leftrightarrow |x| > \sqrt[4]{16} \Leftrightarrow x < -2 \text{ o bien } > 2$$
    El dominio es:
    $$Dom(f) = (-\infty, -2) \cup (2, +\infty)$$
    que es simétrico respecto del origen. Además:
    $$f(-x) = \ln((-x)^4-16) = \ln(x^4-16) = f(x)$$
}

\blueBox{Definición - Función impar}{
    Sea una función definida en un conjunto simétrico respecto del origen. $f$ es impar si:
    $$f(-x) = -f(x) \,\,\, \forall x \in Dom(f)$$
    Las gráficas de las funciones impares son simétricas respecto del centro de coordenadas. 
    \tcblower
    \textbf{Ejemplo}: La función $f: \R \rightarrow \R / f(x) = x^3$ es una función impar:
    $$f(-x) = (-x)^3 = -x^3 = -f(x)$$
}

\section{Traslaciones verticales y horizontales}
Las gráficas de una función se pueden trasladar verticalmente o horizontalmente. Esto se hace sumando o restando un número a la variable independiente o a la variable dependiente.
\vspace{10pt}\\
Sean $f$ y $g(x) = x+k$, con $k \in \R$ dos funciones para que se pueda definir $f \circ g$ y $g \circ f$:
    $$(g \circ f)(x) = g(f(x)) = f(x) + k$$
Es una traslación vertical de $k$ unidades (hacia arriba si $k > 0$, o hacia abajo si $k < 0$)
    $$(f \circ g)(x) = f(g(x)) = f(x+k)$$
Es una traslación horizontal de $k$ unidades (hacia la derecha si $k < 0$, o hacia la izquierda si $k > 0$)